\subsection{Runtime Bounds}

We can split the runtime of a program into two parts:
\begin{equation}
    T_1 = W_s + W_p
\end{equation}
$W_s$ indicates the runtime of the non-parallelizable, serial part.
$W_p$ is the runtime of the part that is parallelizable.

\subsubsection{Amdahl's law}

In practice, there is always a serial part of the program.
The serial part imposes a lower bound on the runtime of the program and an upper bound on the maximally achievable speedup.
This is known as Amdahl's law.
\begin{align*}
    T_c &\geq W_s + \frac{W_p}{c} \\
    S_c &\leq \frac{T_1}{W_s + \frac{W_p}{c}} = \frac{1}{f + \frac{1-f}{c}}
\end{align*}
where $f$ is the fraction of the serial runtime to the full runtime:
\begin{equation*}
    f = \frac{W_s}{T_1} = 1 - \frac{W_p}{T_1}
\end{equation*}

So our maximally achievable speedup is
\begin{equation}
    S_\infty \leq \frac{1}{f}
\end{equation}

Amdahl's law shows that minimizing the serial part of a program is a very important task in parallel programming and has a big effect on which speedups are possible.
We can now also see that we can never reach perfect linear scalability.
Instead, our speedup will eventually saturate and adding more cores will no longer give any significant speedup.
Therefore, in practice we usually only consider scalability up to some maximum amount of cores $c^*$.

\subsubsection{Gustafson's law}

Amdahl's law illustrated the importance of the ratio between $W_s$ and $W_p$.
We concluded that minimizing $W_s$ is crucial for achieving good scalability.
However, we can also conclude that increasing $W_p$ has the same effect.
In particular, we can increase the workload in parallel regions of our program without it having an effect on our runtime, as long as we have enough cores available.
This principle is known as Gustafson's law.
