\subsection{Speedup and Scalability}

How does the runtime of a program improve when we increase the number of cores?

\begin{definition}{Speedup}
    Let $T_p$ be the execution time of a program running on $p$ cores.
    The \emph{speedup} $S_p$ is defined as
    $$S_p = \frac{T_1}{T_p}$$
\end{definition}

We can group the speedup into three categories:
\begin{itemize}
    \item $S_p = p$: linear (perfect) speedup
    \item $S_p < p$: Sub-linear speedup, usual case
    \item $S_p > p$: Super-linear speedup, something is probably wrong
\end{itemize}

\emph{Scalability} is an informal term to describe that a program has a smaller runtime when the number of cores increase.
A program scales linearly if we have a linear speedup.
