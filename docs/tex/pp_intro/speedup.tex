\subsection{Speedup and Scalability}

How does the runtime of a program improve when we increase the number of cores?

\begin{definition}{Speedup}
    Let $T_c$ be the execution time of a program running on $c$ compute cores.
    The \emph{speedup} $S_c$ is defined as
    \begin{equation}
        S_c = \frac{T_1}{T_c}
    \end{equation}
\end{definition}

We can group the speedup into three categories:
\begin{itemize}
    \item $S_c = c$: linear (perfect) speedup
    \item $S_c < c$: Sub-linear speedup, usual case
    \item $S_c > c$: Super-linear speedup, something is probably wrong
\end{itemize}

Super-linear speedup can only be achieved, if parallelization also improves single core performance of at least some part of the program.

\emph{Scalability} is an informal term to describe that a program has a smaller runtime when the number of cores increase.
A program scales linearly if we have a linear speedup.
