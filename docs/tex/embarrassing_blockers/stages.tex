\subsection{Serial Stages of Computation in Dask}

In the \href{https://docs.dask.org/en/stable/phases-of-computation.html}{dask documentation} one can find an overview on what is happening behind the scenes when using dask for parallel computation.
Only the "Execution" phase runs in parallel.
All the other stages are run sequentially and are therefore part of the serial overhead.
Finch can profile the runtimes of those stages, except for the "Graph Communication" and the "Scheduling" stage.
We can therefore get an estimation about the fraction $f$ of the serial runtime in our program.

\paragraph{Graph Construction}
Here we capture everything that prepares the computation.
Finch runs the actual implementation of the operator in this stage, which is supposed to construct the dask graph.

\paragraph{Graph optimization}
Dask performs some optimization heuristics on the graph to decrease the runtime.
Finch profiles this separately.

\paragraph{Graph serialization}
Dask distributed (which we use) runs the scheduler in a separate process than the client.
To communicate with other processes, the scheduler (as well as the workers) are implemented as servers.
In order to send the computation graph to the scheduler, dask must therefore serialize the graph before sending it.
Finch measures the runtime of this graph serialization on the client side.

All the above stages of computation happen sequentially on the client, which is the process from which we launch the whole computation.
Unfortunately this is not the complete story.
There is still some serial overhead happening after sending the serialized computation graph to the scheduler.
They can be split up into the following parts:
\begin{itemize}
    \item Graph Communication: Sending the serialized graph from the client to the scheduler over the network.
    \item Graph Deserialization: Deserializing the graph sent by the client on the scheduler.
    \item Scheduling: Managing individual tasks and sending them to the workers. This happens before and during the parallel computation phase.
\end{itemize}

Unfortunately, we cannot profile the above overheads.
But luckily, we can usually assume that their runtime reduces along with runtime reductions in serial stages which we can profile.